\documentclass[12pt, leqno]{article} %% use to set typesize
\usepackage{fancyhdr}
\usepackage[sort&compress]{natbib}
\usepackage[letterpaper=true,colorlinks=true,linkcolor=black]{hyperref}

\input{commonm}

\newcommand{\hdr}[2]{
  \pagestyle{fancy}
  \lhead{Bindel, Fall 2019}
  \rhead{Matrix Computation}
  \fancyfoot{}
  \begin{center}
    {\large{\bf HW for #1}} \\ (due: #2)
  \end{center}
  \lstset{language=matlab,columns=flexible}
}


\begin{document}

\hdr{2019-10-21}

\section{Eigenvalue problems}

An eigenvalue $\lambda \in \bbC$ of a matrix $A \in \bbC^{n \times n}$
is a value for which the equations $A v = v \lambda$ and $w^* A = \lambda w^*$
have nontrivial
solutions (the eigenvectors $w^*$ and $v$).  Together, $(\lambda, v)$ forms an
eigenpair and $(\lambda, v, w^*)$ forms an eigentriple.
An eigenvector is a basis for a one-dimensional invariant
subspace: that is, $A$ maps anything multiple of $v$ to some other
multiple of $v$.  More generally, a matrix $V \in \bbC^{n \times m}$
spans an invariant subspace if $AV = VL$ for some
$L \in \bbC^{n \times m}$.

Associated with any square $A$, we can write a matrix $Q$ whose
columns form an orthonormal basis for nested invariant subspaces of
$A$; that is, the first $k$ columns of $Q$ form a $k$-dimensional
invariant subspace of $A$.  This structure of nested invariant
subspaces gives us that
\[
  AQ = QT,
\]
where $T$ is an upper triangular matrix.  The factorization
\[
  A = QTQ^*
\]
is a {\em Schur factorization}.  Most of next week will be devoted to
methods to compute Schur factorizations (or parts of Schur
factorizations).  The Schur factorization is nearly as versatile as,
and is far more numerically stable than, the {\em Jordan canonical form}
\[
  A V = V J.
\]
where $J$ is a block diagonal matrices with
{\em Jordan} blocks of the form
\[
  J_{\lambda} =
  \begin{bmatrix}
    \lambda & 1  \\
            & \ddots & \ddots \\
            &        & \lambda & 1 \\
            &        &         & \lambda
  \end{bmatrix}.
\]

The {\em algebraic multiplicity} of an eigenvalue $\lambda$ is the
number of times it appears on the diagonal of the Jordan form, or the
number of times the factor $z-\lambda$ divides the characteristic
polynomial $\det(A-zI)$.  The {\em geometric multiplicity} is given by
the number of Jordan blocks associated to $\lambda$, or by the
dimension of the null space of $(A-\lambda I)$.  In general, there is
exactly one eigenvector of $A$ for each Jordan block, and the
eigenvectors form a basis iff $A$ is diagonalizable -- that is, if $A$
has only 1-by-1 Jordan blocks and all geometric and algebraic
multiplicities match.  The diagonalizable matrices form a dense set in
$\bbC^{n \times n}$, a fact which is often convenient in proofs (since
an argument for the diagonalizable case together with a continuity
argument often yields a general solution).  This fact also explains
part of why the Jordan canonical form is annoying for numerical work:
if every matrix is an arbitrarily small perturbation of something
diagonalizable, then the Jordan form is discontinuous as a function of
$A$!  Even among the diagonalizable matrices, though, the eigenvector
decomposition is often overrated for computational purposes.  Poor
conditioning of the eigenvector basis can make diagonalization a
numerically unstable business, and most computations that are naively
formulated in terms of an eigenvector basis can equally well be
formulated in terms of Schur basis.

In {\em generalized eigenvalue problems}, we ask for nontrivial
solutions to
\[
  (A-\lambda B)v = 0.
\]
There are also {\em nonlinear eigenvalue problems}, which show up
in my research but which we will not talk about in class.  In
addition to these variants on the eigenvalue problem, there are
also many different factors that affect the how we choose algorithms.
Is the problem...
\begin{enumerate}
\item
  nonsymmetric or symmetric?
\item
  standard or generalized?
\item
  to find all eigenvalues or just a few?
\item
  to compute eigenvectors, invariant subspaces, or just eigenvalues?
\end{enumerate}
For different answers to these questions, there are different ``best''
choices of algorithm.  For the next week or two, we will focus
specifically on the problem of computing eigenpairs, invariant
subspaces, and Schur forms for nonsymmetric matrices.  We will only
briefly touch on the special case of the symmetric problem, which has
so much more mathematical structure that it is treated almost entirely
differently from the nonsymmetric case.

\subsection{The 2-by-2 case: some illustrative examples}

Many of the salient features that occur in general eigenvalue
problems can be illustrated with the 2-by-2 matrix
\[
  A = \begin{bmatrix} a & b \\ c & d \end{bmatrix}.
\]
Finding an eigenvalue is equivalent to finding a root
of the characteristic polynomial:
\begin{align*}
  p(z)
    &= \det(A-z I) = (a-z)(d-z) - bc \\
    &= z^2 - (a + d) z + (ad-bc).
\end{align*}
If the roots of the characteristic polynomial are $\lambda_1$ and
$\lambda_2$, then we have
\begin{align*}
  p(z) &= (z-\lambda_1)(z-\lambda_2) \\
       &= z^2 - (\lambda_1 + \lambda_2) z + \lambda_1 \lambda_2.
\end{align*}
We recognize the second coefficient in the
characteristic polynomial as minus the trace $a + d = \lambda_1 +
\lambda_2$.  The constant coefficient is the determinant
$ad-bc = \lambda_1 \lambda_2$.  Both these coefficients can be
seen as functions of the eigenvalues, but both can be computed
efficiently without referring to the eigenvalues explicitly.

Now suppose we choose some fixed $\lambda \in \bbC$ and look at the
2-by-2 matrices for which $\lambda$ is an eigenvalue.  If we just want
$\lambda$ to be {\em an} eigenvalue, we must satisfy one scalar
equation: $p(\lambda) = 0$.  To find matrices for which $\lambda$ is a
{\em double} eigenvalue, we must satisfy the additional constraint
$a+d = 2\lambda$.  And there is only one 2-by-2 matrix for which $\lambda$ is
a double eigenvalue with {\em geometric} multiplicity 2: $A = \lambda I$.
Put differently, the set of 2-by-2 matrices for which $\lambda$ is an
eigenvalue has codimension 1 (i.e. it is described by one scalar
constraint); the set of 2-by-2 matrices for which $\lambda$ is an eigenvalue
with algebraic multiplicity 2 has codimension 2; and the set of 2-by-2
matrices for which $\lambda$ is an eigenvalue with geometric multiplicity
2 has codimension 3.

More generally, we can say that among general complex $n$-by-$n$ matrices,
the existence of {\em some} multiple eigenvalue is a
codimension 1 phenomena (somewhat rare in general); and the existence
of an eigenvalue with {\em geometric} multiplicity greater than 1 is
a codimension 3 phenomena (very rare in general).  Of course, things
change if we consider structured matrices.  For example, in symmetric
matrices the  algebraic and geometric multiplicities of all eigenvalues
are the same.

\subsection{The symmetric case}

In general, a real matrix can have complex eigenvalues (though in
conjugate pairs), and it may or may not have a basis of eigenvectors.
In the case of real {\em symmetric} matrices ($A = A^T$), we have
much more structure: namely,
\begin{itemize}
\item All the eigenvalues are real.
\item There is a complete {\em orthonormal} basis of eigenvectors.
\end{itemize}
To see the former, observe that if $(v, \lambda)$ is an eigenpair
and $\|v\| = 1$ then
\[
  \lambda = v^* A v = \bar{v^* A v} = \bar{\lambda},
\]
which implies that $\lambda$ is real.  To see that eigenvectors
associated with different eigenvalues must be orthogonal, note
that if $(v,\lambda)$ and $(u,\mu)$ are eigenpairs with $\lambda \neq
\mu$, then
\[
v^* A u = \begin{cases}
  (A v)^* u = \lambda v \cdot u \\
  v^* (Au) = \mu v \cdot u
\end{cases}
\]
and the only way for these to be the same is if $v \cdot u = 0$.
Combining these two facts about the symmetric eigenvalue problem,
we usually write the standard decomposition
\[
  A = Q \Lambda Q^T
\]
where $Q$ is an orthogonal matrix of eigenvalues
and $\Lambda$ is the corresponding diagonal matrix of eigenvalues.

We often use symmetric matrices to represent quadratic forms,
and this is one reason why symmetric eigenvalue problems are
so common.  If $A = Q \Lambda Q^T$, then we can define $w = Q^T v$
to get the expression
\[
  v^* A v = \sum_{i=1}^n w_i^2 \lambda_i.
\]
If $\|v\|^2 = 1$ (implying $\|w\|^2 = 1$), then we can see $v^* A v$ is a
weighted average of the eigenvalues of $A$.  Hence, the minimum or
maximum of $v^* A v$ over all unit length vectors gives the largest
and smallest of the eigenvalues of $A$; and, more generally, the
eigenvalues are stationary points of $v^* A v$ subject to the
constraint $\|v\|^2 = 1$.  Sometimes we prefer to work with all nonzero
vectors rather than vectors with unit length, and hence define the
{\em Rayleigh quotient}
\[
  \rho_A(v) = \frac{v^* A v}{v^* v};
\]
this ratio plays a central role in the theory of the symmetric eigenproblem.

\section{Why eigenvalues?}

I spend a lot of time thinking about eigenvalue problems.  In part,
this is because I look for problems that can be solved via
eigenvalues.  But I might have fewer things to keep me out of trouble
if there weren't so many places where eigenvalue analysis is useful!
The purpose of this lecture is to tell you about a few applications of
eigenvalue analysis, or perhaps to remind you of some applications
that you've seen in the past.

\subsection{Nonlinear equation solving}

The eigenvalues of a matrix are the roots of the characteristic
polynomial
\[
  p(z) = \det(zI-A).
\]
One way to compute eigenvalues, then, is to form the characteristic
polynomial and run a root-finding routine on it.  In practice, this is
a terrible idea, if only because the root-finding problem is often far
more sensitive than the original eigenvalue problem.  But even if
sensitivity were not an issue, finding {\em all} the roots of a
polynomial seems like a nontrivial undertaking.  Iterations like
Newton's method, for example, only converge locally.  In fact, the
{\tt roots} command in MATLAB computes the roots of a polynomial
by finding the eigenvalues of a corresponding {\em companion matrix}
with the polynomial coefficients on the first row, ones on the first
subdiagonal, and zeros elsewhere:
\[
  C = \begin{bmatrix}
    c_{d-1} & c_{d-2} & \ldots & c_1 & c_0 \\
    1   & 0   & \ldots & 0 & 0 \\
    0   & 1   & \ldots & 0 & 0 \\
    \vdots   & \vdots    & \ddots &     \vdots & \vdots \\
    0   & 0   & \ldots & 1 & 0
    \end{bmatrix}.
\]
The characteristic polynomial for this matrix is precisely
\[
  \det(zI-C) = z^d + c_{d-1} z^{d-1} + \ldots + c_1 z + c_0.
\]

There are some problems that connect to polynomial root finding, and
thus to eigenvalue problems, in surprising ways.  For example, the
problem of finding ``optimal'' rules for computing integrals
numerically (sometimes called Gaussian quadrature rules) boils down to
finding the roots of orthogonal polynomials, which can in turn be
converted into an eigenvalue problem; see, for example, ``Calculation
of Gauss Quadrature Rules'' by Golub and Welsch ({\em Mathematics of
Computation}, vol 23, 1969).

More generally, eigenvalue problems are one of the few examples I have
of a nonlinear equation where I can find {\em all} solutions in
polynomial time!  Thus, if I have a hard nonlinear equation to solve,
it is very tempting to try to massage it into an eigenvalue problem,
or to approximate it by an eigenvalue problem.

\subsection{Optimization}

As we noted before, the symmetric eigenvalue problem has an
interpretation in terms of optimization of a quadratic form over
unit length vectors.  More generally, one can look at generalized
eigenvalue problems in terms of optimization of a ratio of quadratic
forms.  We now discuss some applications where this interpretation
is useful.

Recall that the matrix 2-norm is defined as
\[
  \|A\|_2 = \max_{x \neq 0} \frac{\|Ax\|}{\|x\|} = \max_{\|x\| = 1} \|Ax\|.
\]
Taking squares and using the monotonicity of the map
$z \rightarrow z^2$ for non-negative arguments, we have
\[
  \|A\|_2^2 = \max_{\|x\|^2 = 1} \|Ax\|^2 = \max_{x^T x = 1} x^T A^T A x.
\]
The $x$ that solves this constrained optimization problem must be a
stationary point for the augmented Lagrangian function
\[
  L(x,\lambda) = x^T A^T A x - \lambda (x^T x - 1),
\]
i.e.
\begin{align*}
  \nabla_x L(x,\lambda) &= 2(A^T A x - \lambda x) = 0 \\
  \nabla_\lambda L(x,\lambda) &= x^T x - 1 = 0.
\end{align*}
These equations say that $x$ is an eigenvector of $A^T A$
with eigenvalue $\lambda$.  The largest eigenvalue of $A^T A$ is
therefore $\|A\|_2^2$.

More generally, if $H$ is any Hermitian matrix, the {\em Rayleigh quotient}
\[
  \rho_H(v) = \frac{v^* H v}{v^* v}
\]
has stationary points exactly when $v$ is an eigenvector of $H$.
Optimizing the Rayleigh quotient is therefore example of a
{\em non-convex} global optimization problem that I know how to solve in
polynomial time.  Such examples are rare, and so it is tempting to try
to massage other nonconvex optimization problems so that they look
like Rayleigh quotient optimization, too.

To give an example of a nonconvex optimization that can be usefully
approximated using Rayleigh quotients, consider the superficially
unrelated problem of graph bisection.  Given an undirected graph $G$
with vertices $V$ and edges $E \subset V \times V$, we want to find a
partition of the nodes into two equal-size sets such that few edges go
between the sets.  That is, we want to write $V$ as a disjoint union
$V = V_1 \cup V_2$, $|V_1| = |V_2|$, such that the number of edges cut
$|E \cap (V_1 \times V_2)|$ is minimized.  Another way to write the
same thing is to label each node $i$ in the graph with $x_i \in \{+1,
-1\}$, and define $V_1$ to be all the nodes with label $+1$, $V_2$ to
be all the nodes with label $-1$.  Then the condition that the two
sets are the same size is equivalent to
\[
  \sum_i x_i = 0,
\]
and the number of edges cut is
\[
  \frac{1}{4} \sum_{(i,j) \in E} (x_i-x_j)^2
\]
We can rewrite the constraint more concisely as $e^T x = 0$, where $e$
is the vector of all ones; as for the number of edges cut, this is
\[
  \mbox{edges cut} = \frac{1}{4} x^T L x
\]
where the {\em graph Laplacian} $L$ has the node degrees on the
diagonal and $-1$ in off-diagonal entry $(i,j)$ iff there is an edge
from $i$ to $j$.

Unsurprisingly, the binary quadratic programming problem
\[
  \mbox{minimize } x^T L x \mbox{ s.t. } e^T x = 0 \mbox{ and } x \in
  \{ +1, -1 \}^n
\]
is NP-hard, and we know of no efficient algorithms that are guaranteed
to work for this problem in general.  On the other hand, we can
{\em relax} the problem to
\[
  \mbox{minimize } v^T L v \mbox{ s.t. } e^T v = 0 \mbox{ and } \|v\|^2
  = n, v \in \bbR^n,
\]
and this problem is an eigenvalue problem: $v$ is the eigenvector
associated with the smallest positive eigenvalue of $L$, and $v^T L v$
is $n$ times the corresponding eigenvalue.  Since the constraint in
the first problem is strictly stronger than the constraint in the
second problem, $n \lambda_2(L)$ is in fact a lower bound on the
smallest possible cut size, and the sign pattern of $v$ often provides
a partition with a small cut size.  This is the heart of {\em spectral
  partitioning} methods.

\subsection{Dynamics}

Eigenvalue problems come naturally out of separation of variables
methods, and out of transform methods for the dynamics of discrete or
continuous linear time invariant systems, including examples from
physics and from probability theory.  They allow us to analyze
complicated high-dimensional dynamics in terms of simpler,
low-dimensional systems.  We consider two examples: separation of
variables for a free vibration problem, and convergence of a
discrete-time Markov chain.

\subsubsection{Generalized eigenvalue problems and free vibrations}

One of the standard methods for solving differential equations is
{\em separation of variables}.  In this approach, we try to write
special solutions as a product of simpler functions, and then write
the equations that those functions have to satisfy.  As an example,
consider a differential equation that describes the free vibrations of a
mechanical system:
\[
  M \ddot{u} + K u = 0
\]
Here $M \in \bbR^{n \times n}$ is a symmetric positive definite {\em
mass matrix} and $K \in \bbR^{n \times n}$ is a symmetric {\em
stiffness matrix} (also usually positive definite, but not always).
We look for solutions to this system of the form
\[
  u(t) = u_0 \cos(\omega t),
\]
where $u_0$ is a fixed vector.  To have a solution of this form, we
must have
\[
  K u_0 - \omega^2 M u_0 = 0,
\]
i.e. $(\omega^2, u_0)$ is an eigenpair for a {\em generalized}
eigenvalue problem.  In fact, the eigenvectors for this generalized
eigenvalue problem form an $M$-orthonormal basis for $\bbR^n$, and so
we can write {\em every} free vibration as a linear combination of
these simple ``modal'' solutions.

\subsubsection{Markov chain convergence and the spectral gap}

This high-level idea of using the eigenvalue decomposition to
understand dynamics is not limited to differential equations, nor to
mechanical systems.  For example, a {\em discrete-time Markov chain}
on $n$ states is a random process where the state $X_{k+1}$ is a
random variable that depends only on the state $X_k$.  The {\em
  transition matrix} for the Markov chain is a matrix $P$ where
$P_{ij}$ is the (fixed) probability of transitioning to state $i$ from
state $j$, i.e.
\[
  P_{ij} = P\{X_{k+1} = j | X_{k} = i\}.
\]
Let $\pi^{(k)} \in \bbR^n$ be the distribution vector at time $k$,
i.e.
\[
  \pi^{(k)}_i = P\{X_k = i\}.
\]
Then we have the recurrence relationship
\[
  (\pi^{(k+1)})^T = (\pi^{(k)})^T P.
\]
In general, this means that
\[
  (\pi^{(k)})^T = (\pi^{(0)})^T P^k.
\]
Now, suppose the transition matrix $P$ is diagonalizable, i.e. $P = V
\Lambda V^{-1}$.  Then
\[
  P^k = V \Lambda V^{-1} V \Lambda V^{-1} \ldots V \Lambda V^{-1}
      = V \Lambda \ldots \Lambda V^{-1} = V \Lambda^k V^{-1},
\]
and so
\[
  (\pi^{(k)})^T = (\pi^{(0)})^T V \Lambda^k V^{-1}.
\]
An {\em ergodic} Markov chain has one eigenvalue at one, and all the
other eigenvalues are less than one in modulus.  In this case, the row
eigenvector associated with the eigenvalue at one can be normalized so
that the coefficients are all positive and sum to 1.  This normalized
row eigenvector $\pi^{(*)}$ represents the {\em stationary}
distribution to which the Markov chain eventually converges.  To
compute the rate of convergence, one looks at
\[
  \|(\pi^{(k)}-\pi^{(*)})^T\| =
  \|(\pi^{(0)}-\pi^{(*)})^T (V \tilde{\Lambda}^k V^{-1})\| \leq
  \|(\pi^{(0)}-\pi^{(*)})^T \| \, \kappa(V) \|\tilde{\Lambda}\|^k
\]
where $\Lambda = \operatorname{diag}(1, \lambda_2, \lambda_3,
\ldots)$, $|\lambda_i| \geq |\lambda_{i+1}|$, and $\tilde{\Lambda} =
\operatorname{diag}(0, \lambda_2, \lambda_3, \ldots)$.  In most
reasonable operator norms, $|\tilde{\Lambda}|^k = |\lambda_2|^k$,
and so a great deal of the literature on convergence of Markov chains
focuses on $1-|\lambda_2|$, called the {\em spectral gap}.
But note that this bound does not depend on the eigenvalues alone!
The condition number of the eigenvector matrix also plays a role, and
if $\kappa(V)$ is very large, then it may take a long time indeed
before anyone sees the asymptotic behavior reflected by the spectral gap.

\subsection{Deductions from eigenvalue distributions}

In most of our examples so far, we have considered both the
eigenvalues and the eigenvectors.  Now let us turn to a simple example
where the distribution of eigenvalues can be illuminating.

Let $A$ be the adjacency matrix for a graph, i.e.
\[
  A_{ij} = \begin{cases}
    1, & \mbox{ if there is an edge from $i$ to $j$} \\
    0, & \mbox{ otherwise}.
  \end{cases}
\]
Then $(A^k)_{ij}$ is the number of paths of length $k$ from node $i$
to node $j$.  In particular, $(A^k)_{ii}$ is the number of cycles of
length $k$ that start and end at node $i$, and
$\operatorname{trace}(A^k)$ is the total number of length $k$ cycles
starting from any node.  Recalling that the trace of a matrix is the
sum of the eigenvalues, and that the eigenvalues of a matrix power are
the power of the eigenvalues, we have that
\[
  \mbox{\# paths of length $k$} = \sum_{i} \lambda_i(A)^k,
\]
where $\lambda_i(A)$ are the eigenvalues of $A$; and asymptotically,
the number of cycles of length $k$ for very large $k$ scales like
$\lambda_1(A)^k$, where $\lambda_1(A)$ is the largest eigenvalue of
the matrix $A$.

While the statement above deals only with eigenvalues and not with
eigenvectors, we can actually say more if we include the eigenvector;
namely, if the graph $A$ is irreducible (i.e. there is a path from
every state to every other state), then the largest eigenvalue
$\lambda_1(A)$ is a real, simple eigenvalue, and asymptotically the number
of paths from any node $i$ to node $j$ scales like the $(i,j)$ entry
of the rank one matrix
\[
  \lambda_1^k v w^T
\]
where $v$ and $w$ are the column and row eigenvectors of $A$
corresponding to the eigenvalue $\lambda_1$, scaled so that $w^T v = 1$.

\end{document}
