\documentclass[12pt, leqno]{article} %% use to set typesize
\usepackage{fancyhdr}
\usepackage[sort&compress]{natbib}
\usepackage[letterpaper=true,colorlinks=true,linkcolor=black]{hyperref}

\input{commonm}

\newcommand{\hdr}[2]{
  \pagestyle{fancy}
  \lhead{Bindel, Fall 2019}
  \rhead{Matrix Computation}
  \fancyfoot{}
  \begin{center}
    {\large{\bf HW for #1}} \\ (due: #2)
  \end{center}
  \lstset{language=matlab,columns=flexible}
}

\newcommand{\calK}{\mathcal{K}}

\begin{document}

\hdr{2019-12-06}

\section{Lanczos and Arnoldi eigensolvers}

The standard ingredients in all the subspace methods we have
described so far are a choice of an approximation subspace
(usually a Krylov subspace) and a method for choosing an approximation
from the space.  In the most common methods for large-scale eigensolvers,
one uses a Krylov subspace together with a Bubnov-Galerkin condition
for choosing approximate eigenpairs; that is, we choose $v \in \mathcal{V}$
such that
\[
  r = (A-\hat{\lambda} I) \hat{v} \perp \mathcal{V}.
\]
In the symmetric case, this is equivalent to finding a constrained
stationary point of the Rayleigh quotient (i.e.~a $\hat{v}$ such that
directional derivatives of $\rho_A(\hat{v})$ are zero for any direction
in the space).  This approximation scheme is known as the {\em
Rayleigh-Ritz} method, and approximate eigenvectors and eigenvalues
obtained in this way are often called {\em Ritz vectors} and {\em Ritz
values}.

In the Lanczos method for the symmetric eigenvalue problem,
we compute the Lanczos decomposition
\[
  AQ_m = Q_m T_m + \beta_m q_{m+1} e_m^T,
\]
and use it to compute the residual relation for an approximate pair
$(\mu, Q_m y)$ by
\[
  r = (A-\mu I) Q_m y = Q_m (T_m-\mu I) y + \beta_m q_{m+1} e_m^T y.
\]
The condition $Q_m^T r = 0$ gives us the {\em projected} problem
\[
  (T_m-\mu I) y = 0;
\]
if we satisfy this condition, we have
\[
  r = \beta_m q_{m+1} y_m
\]
and the residual norm (in the 2-norm) is $|\beta_m y_m|$.  Generalizing,
if we compute the eigendecomposition
\[
  T_m = Y \Theta Y^T,
\]
we have the collected approximations $Z = Q_m Y_m$ with residuals
\[
  \|Az_k-z_k \theta_k\|_2 = |\beta_m| |e_m^T y_k|.
\]
This is useful because, as we discussed before, in the symmetric case
a small residual error implies a small distance to the closest eigenvalue.
This is also useful because the residual error can be computed with no
further matrix operations --- we need only to look at quantities that
we would already compute in the process of obtaining the tridiangonal
coefficients and the corresponding Ritz values.

The Arnoldi method for computing approximate eigenpairs similarly uses
the Galerkin condition together with the Arnoldi decomposition to
express an approximate partial Schur form.  From the decomposition
\[
  AQ_m = Q_m H_m + h_{m+1,m} q_{m+1} e_m^T,
\]
we write a subspace residual for $(Q_m Y, T)$ as
\[
  R = AQ_m Y - Q_m Y T = Q_m (H_m Y - Y T) + h_{m+1,m} q_{m+1} e_m^T.
\]
Forcing $Q_m^T R = 0$ gives the projected problem
\[
  H_m Y = Y T,
\]
i.e.~we seek a Schur decomposition of the (already Hessenberg)
matrix $H_m$.

There are three main issues with the Lanczos and Arnoldi methods
that we need to address in practical situations.
\begin{enumerate}
\item
  We must deal with forward instability, particularly in the case of the
  Lanczos method.  Unless we are careful to maintain orthogonality
  between the computed Lanczos basis vectors, the method derails.  The
  result is not that we get bad approximate eigenpairs; indeed, the
  forward instability is intimately tied to the very thing we want,
  which is convergence of eigenpairs.  The real problem is that we get
  the same eigenpairs over and over again, a phenomenon known as
  ``ghost'' eigenvalue approximations.  We deal with this issue by
  careful re-orthogonalization (selective or complete).
\item
  Because of the cost of storing a Krylov basis and maintaining its
  orthogonality, we typically only want to approximate a few eigenpairs
  at a time.
\item
  The Krylov subspace generated by $A$ and some random start
  vector contains iterates of the power method applied to any $A$
  (or to $A-\sigma I$ for any shift $\sigma$ --- the Krylov subspace
  is shift-invariant).  This is at least as good as power iteration
  for approximating the extremal parts of the spectrum, and we can
  use the same Chebyshev-based games we discussed before to
  give concrete (though typically pessimistic) convergence bounds.  But if
  eigenvalues cluster, or if we are interested in eigenvalues that
  are not at the edge of the spectrum, then the convergence in theory
  and in practice can be painfully slow.
\end{enumerate}
We address these issues with two basic techniques, both of which
we have already seen in other contexts: {\em spectral transformation}
and {\em restarting}.

\section{Spectral transformation}

We have dealt with the notion of spectral transformation before,
when we discussed the power iteration.  The idea of spectral
transformation is to work not with $A$, but with some rational
$f(A)$ where $f$ maps the eigenvalues of interest to the outside
of the spectrum.  Usually $f$ is a rational function; common
examples include
\begin{itemize}
\item {\em Shift-invert}: $f(z) = (z-\sigma)^{-1}$.  Favors
  eigenvalues close to the shift $\sigma$.
\item {\em Cayley}: $f(z) = (\sigma-z) (\sigma +z)^{-1}$.  This maps
  the left half plane to the interior of the unit circle and the
  right half plane to the exterior; it is commonly used in stability
  analysis.
\item {\em Polynomial}: Just what it sounds like.
\end{itemize}
In general, the shifted linear solves needed to carry out rational spectral
transformations (e.g. shift-invert and Cayley) must be computed to
rather high accuracy.  Hence, we favor sparse direct methods.  An alternate
approach, similar to what we say when we briefly considered flexible GMRES,
is to break out of the confines of using a Krylov subspace; the most
popular variant here is the {\em Jacobi-Davidson} method.

\subsection{Jacobi-Davidson}

The {\em Jacobi-Davidson} iteration is an alternative subspace-based
large-scale eigenvalue solver that does not use Krylov subspaces.
Instead, one builds a subspace via steps of an inexact Newton iteration
on the eigenvalue equation.  Given an approximate eigenpair $(\theta,u)$
where $\theta$ is the Rayleigh quotient, we seek a correction $s \perp u$
so that
\[
  A(u+s) = \lambda (u+s).
\]
Rewriting this in terms of $r = (A-\theta I) u$, we have for any
approximate $\tilde{\lambda}$ to the desired eigenvalue
\[
  (A-\tilde{\lambda} I) s = -r + (\lambda-\theta) u + (\lambda-\tilde{\lambda}) s.
\]
Using the desiderata that $u^* s = 0$ and the fact that $(I-uu^*) u = 0$,
we obtain the correction equation
\[
  (I-uu^*) (A-\tilde{\lambda} I)(I-uu^*) s = -r, \quad \mbox{where } s \perp u.
\]
The method proceeds by at each step solving the correction equation
approximately and extending the subspace by a new direction $s$.
One then seeks an approximate eigenpair from within the subspace.

In addition to a proper choice of subspaces, one needs a method to
extract approximate eigenvectors and eigenvalues.  This is particularly
important for approximating interior eigenvalues, as the standard
Rayleigh-Ritz approach may give poor results there.  One possible
method is to use the {\em refined Ritz} vector, which is obtained by
minimizing the residual over all candidate eigenvectors associated
with an approximate eigenvalue $\tilde{\lambda}$.  The refined Ritz
vector may then be plugged into the Rayleigh quotient to obtain a
new eigenvector.  Another method is the {\em harmonic Rayleigh-Ritz}
approach, which for eigenvalues near a target $\tau$ employs the
condition
\[
  (A-\tau I)^{-1} \tilde{u} - (\tilde{\theta}-\tau)^{-1} \tilde{u} \perp \mathcal{V}.
\]
We again usually use the Rayleigh quotient from the harmonic Ritz vector
rather than the harmonic Ritz value $\tilde{\theta}$.

\end{document}
