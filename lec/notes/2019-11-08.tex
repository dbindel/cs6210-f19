\documentclass[12pt, leqno]{article} %% use to set typesize
\usepackage{fancyhdr}
\usepackage[sort&compress]{natbib}
\usepackage[letterpaper=true,colorlinks=true,linkcolor=black]{hyperref}

\input{commonm}

\newcommand{\hdr}[2]{
  \pagestyle{fancy}
  \lhead{Bindel, Fall 2019}
  \rhead{Matrix Computation}
  \fancyfoot{}
  \begin{center}
    {\large{\bf HW for #1}} \\ (due: #2)
  \end{center}
  \lstset{language=matlab,columns=flexible}
}


\begin{document}

\hdr{2019-11-08}

\section{Sylvester equations}

The {\em Sylvester equation} (or the special case of
the {\em Lyapunov equation}) is a matrix equation of the form
\[
  AX + XB = C
\]
where $A \in \bbR^{m \times m}, B \in \bbR^{n \times n}, B \in \bbR^{m \times n}$, are known,
and $X \in \bbR^{m \times n}$ is to be determined.  The Sylvester
equation has important applications in control theory, and also plays
a prominent role in the theory of several classes of structured matrices.

On the surface of it,
this is a simple system: the expressions $AX$ and $XB$ are just linear
in the elements of $X$, after all.  Indeed, we can rewrite the system
as
\[
  (I \otimes A + B^T \otimes I)
  \operatorname{vec}(X) =
  \operatorname{vec}(C),
\]
where
\[
  \operatorname{\vec}(\begin{bmatrix} x_1 & \ldots & x_n \end{bmatrix}) =
  \begin{bmatrix} x_1 \\ \vdots \\ x_n \end{bmatrix}
\]
is a vector of length $mn$ composed by listing the elements of $X$ in
column-major order, and the {\em Kronecker product} is defined by
\[
  F \otimes G =
  \begin{bmatrix}
    f_{11} G & f_{12} G & \ldots \\
    f_{21} G & f_{22} G & \ldots \\
    \vdots & \vdots & \ddots
  \end{bmatrix}.
\]
Alas, solving this matrix equation by Gaussian elimination would cost
$O((mn)^3)$.  Can we do better?

The {\em Bartels-Stewart} algorithm is a clever approach to the problem
that takes only $O(\max(m,n)^3)$ time.  The key is to compute the
Schur factorizations
\begin{align*}
  A &= U_A T_A U_A^* &
  B &= U_B T_B U_B^*
\end{align*}
from which we obtain
\[
  T_A \tilde{X} + \tilde{X} T_B = \tilde{C}
\]
where $\tilde{X} = U_A^* X U_B$ and $\tilde{C} = U_A^* C U_B$.
Column $j$ of this system of equations can be written as
\[
  (T_A + t_{B,kk} I) \tilde{x}_k =
  \tilde{c}_k - \sum_{j=1}^{k-1} \tilde{x}_j t_{B,jk};
\]
therefore, we can solve each column of $\tilde{x}$ in turn by
a back-substitution procedure that involves a triangular
linear solve.  We only run into trouble if one of these systems
is singular (or nearly so), corresponding to the case where $A$
and $-B$ (nearly) have an eigenvalue in common.

\subsection{Riccati equations}

The Sylvester equation is a {\em linear} matrix equation whose solution
is accelerated via an intermediate eigendecomposition.  The
{\em algebraic Riccati equation} is a {\em quadratic} matrix equation
that also can be expressed via an eigenvalue problem.  The Riccati
equation occurs in optimal control problems, as well as some other places;
for the continuous-time optimal control problem, we would usually write
\[
  A^T X + XA - XBR^{-1} B^T X + Q = 0
\]
where $R$ and $Q$ are spd matrices representing cost functions,
$A$ and $B$ are general square matrices, and we seek a symmetric solution
matrix $X$.

The key to thinking of the Riccati equation via eigenvalues is to write
the left hand side of the equation as a pure quadratic:
\[
  \begin{bmatrix} I \\ X \end{bmatrix}
  \begin{bmatrix} Q & A \\ A^T & -BR^{-1}B^T \end{bmatrix}
  \begin{bmatrix} I \\ X \end{bmatrix} = 0.
\]
We can also characterize this by the relation
\[
  \begin{bmatrix} Q & A \\ A^T & -BR^{-1} B^T \end{bmatrix}
  \begin{bmatrix} I \\ X \end{bmatrix} =
  \begin{bmatrix} 0 & I \\ -I & 0 \end{bmatrix}
  \begin{bmatrix} I \\ X \end{bmatrix},
\]
or, equivalently
\[
  Z = \begin{bmatrix}
    A & -BR^{-1} B^T \\
    -Q & -A^T
  \end{bmatrix}, \quad Z 
  \begin{bmatrix} I \\ X \end{bmatrix} = 0.
\]
That is, we want a specific basis of the null space of
a {\em Hamiltonian matrix}, i.e. a matrix $Z$ such that
\[
  JZ \mbox{ symmetric}, J \equiv \begin{bmatrix} 0 & I \\ -I & 0 \end{bmatrix}.
\]

Hamiltonian eigenvalue problems show up in a surprising variety of
places in addition to optimal control. The theory of eigenvalue problems
for Hamiltonian and skew-Hamiltonian matrices is reasonably well
developed, and the eigenvalues have a special symmetry to them.  There
is now good software for these classes of problems that exploits the
structure --- though not in LAPACK.  The right place to look for these
solvers is in the SLICOT package.

\section{Polynomial eigenvalue problems}

A {\em nonlinear eigenvalue problem} is an equation of the form
\[
  T(\lambda) v = 0
\]
where $T : \bbC \rightarrow \bbC^{n \times n}$ is a matrix-valued
function.  The most common nonlinear eigenvalue problems are
{\em polynomial eigenvalue problems} in which $T$ is a polynomial;
and most common among the polynomial eigenvalue problems are the
{\em quadratic eigenvalue problems}
\[
  (\lambda^2 M + \lambda D + K) u = 0.
\]
As the notation might suggest, one of the natural sources of
quadratic eigenvalue problems is in the analysis of damped
unforced vibrations in mechanical (or other physical) systems.
In this context, $M$, $D$, and $K$ are the {\em mass},
{\em damping}, and {\em stiffness} matrices, and the eigenvalue
problem arises from the search for special solutions to the
equation
\[
  M\ddot{x} + D\dot{x} + Kx = 0
\]
where $x(t) = u \exp(\lambda t)$.  We note that the mass matrix
is often symmetric and positive definite; in this case, we can
apply a change of variables to convert to a problem in which the
leading term involves an identity matrix.  We will assume this case
for the remainder of our discussion.

When studying the solution of higher-order differential equations,
a standard trick is to put the system into first-order form by
introducing auxiliary variables for derivatives.  For example,
we would put our model second-order unforced vibration equation
into first-order form by introducing the variable $v = \dot{x}$;
then (assuming $M = I$), we have
\[
  \begin{bmatrix} \dot{v} \\ \dot{x} \end{bmatrix} =
  \begin{bmatrix} -D & -K \\ I & 0 \end{bmatrix}
  \begin{bmatrix} v \\ x \end{bmatrix}.
\]
Similarly, we can convert the quadratic eigenvalue problem into
a standard linear eigenvalue problem by introducing $w = \lambda u$;
then
\[
  \lambda \begin{bmatrix} w \\ u \end{bmatrix} =
  \begin{bmatrix} -D & -K \\ I & 0 \end{bmatrix}
  \begin{bmatrix} w \\ u \end{bmatrix}.
\]
This process of converting a quadratic (or higher-order polynomial)
eigenvalue problem into a linear eigenvalue problem in a higher-dimensional
space is called {\em linearization} (a somewhat unfortunate term, but the
standard choice).  There are many ways to define the auxiliary variables,
and hence many ways to linearize a polynomial eigenvalue problem; the version
we have described is the {\em companion} linearization.  Different
linearizations are appropriate to polynomial eigenvalue problems with
different structure.

More generally, a ``genuinely'' nonlinear eigenvalue involves
a matrix $T(\lambda)$ that depends on the spectral parameter $\lambda$
as a more general non-rational function.  Typically, we restrict our
attention to functions that are complex-analytic in some domain of
interest; these arise naturally in many applications, particularly
in problems involving delay, radiation, and similar effects.  One
thread in my own research has been to extend some of the theory we
have for the standard eigenvalue problem --- results like Gershgorin
and Bauer-Fike --- to this more general nonlinear case.

\section{Pseudospectra}

We conclude our discussion of eigenvalue-related ideas by revisiting the
perturbation theory for the nonsymmetric eigenvalue problem from a
somewhat different perspective.  In the symmetric case, if $A-\hat{\lambda} I$
is nearly singular (i.e. $(A-\hat{\lambda} I) \hat{x} = r$ where
$\|r\| \ll \|A\|\|\hat{x}\|$), then $\hat{\lambda}$ is close to one of the
eigenvalues of $A$.  But in the nonsymmetric case, $A-\hat{\lambda I}$ may
become quite close to singular even though $\hat{\lambda}$ is quite far
from any eigenvalues of $A$.  The approximate null vector of $A-\hat{\lambda} I$
is sometimes called a {\em quasi-mode}, and dynamical systems defined via
such a matrix $A$ are often characterized by long-lived transient dynamics
that are well-described in terms of such quasi-modes.

In order to describe quasi-modes and long-lived transients, we need a
systematic way of thinking about ``almost eigenvalues.''  This leads
us to the idea of the $\epsilon$-{\em pseudospectrum}:
\[
  \Lambda_{\epsilon}(A) = \{z \in \bbC : \|(A-zI)^{-1}\| \geq \epsilon^{-1}\}.
\]
This is equivalent to
\[
  \Lambda_{\epsilon}(A) = \{ z \in \bbC : \exists E \mbox{ s.t. } \|E\| < \epsilon \mbox{ and }(A+E-zI) \mbox{ singular} \},
\]
or, when the norm involved is the operator $2$-norm,
\[
  \Lambda_{\epsilon}(A) = \{ x \in \bbC : \sigma_{\min}(A-zI) < \epsilon \}.
\]
There is a great deal of beautiful theory involving pseudospectra; as a
guide to the area, I highly recommend {\em Spectra and Pseudospectra}
by Mark Embree and Nick Trefethen.

\end{document}
