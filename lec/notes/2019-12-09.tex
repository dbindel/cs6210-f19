\documentclass[12pt, leqno]{article} %% use to set typesize
\usepackage{fancyhdr}
\usepackage[sort&compress]{natbib}
\usepackage[letterpaper=true,colorlinks=true,linkcolor=black]{hyperref}

\input{commonm}

\newcommand{\hdr}[2]{
  \pagestyle{fancy}
  \lhead{Bindel, Fall 2019}
  \rhead{Matrix Computation}
  \fancyfoot{}
  \begin{center}
    {\large{\bf HW for #1}} \\ (due: #2)
  \end{center}
  \lstset{language=matlab,columns=flexible}
}

\newcommand{\calK}{\mathcal{K}}

\begin{document}

\hdr{2019-12-09}

\section{Big ticket items}

\subsection{Linear algebra and calculus}

\begin{itemize}
\item Linear algebra background (abstract and concrete)
  \begin{itemize}
  \item Vectors, spaces, subspaces, bases
  \item Interpreting matrices: operators, mappings, quadratic forms
  \item Canonical forms
  \end{itemize}
\item Calculus with matrices
  \begin{itemize}
  \item Sensitivity analysis and conditioning
  \item Variational notation for derivatives
  \item Optimization with quadratics
  \item Lagrange multipliers and constraints
  \end{itemize}
\end{itemize}

\subsection{Matrix algebra}

\begin{itemize}
\item Ways to write matrix-matrix products
\item Blocked matrices and blocked algorithms
\item Graph structures: sparse, diagonal, triangular, Hessenberg, etc
\item LA structures: symmetric, skew, orthogonal, etc
\item Other structure: Toeplitz, Hankel, other special matrices
\end{itemize}

\subsection{The big problems}

\begin{align*}
  Ax &= b \\
  \mbox{minimize } \|Ax&-b\|^2 \\
  Ax &= x \lambda
\end{align*}

\subsection{The big factorizations}

\begin{itemize}
\item LU and company ($LDL^T$ and Cholesky)
\item QR (economy and full)
\item SVD (economy and full)
\item Schur factorization
\item Symmetric eigendecomposition
\end{itemize}

\subsection{Iterations}

\begin{itemize}
\item Iterative refinement
\item Stationary iterations (Jacobi, Gauss-Seidel, etc)
\item Krylov subspace definition
\item Approximation from a subspace and Galerkin
\item Characterization of CG and GMRES
\end{itemize}

\subsection{Philosophical odds and ends}

\begin{itemize}
\item Identifying the right structure matters a lot
\item We need {\em both} algebra and analysis
\item When you don't know what else to do... eigenvalues or SVD
\item I differentiate five expressions before breakfast!
\end{itemize}

\section{What else?}

There is a lot that I wish I could get to in a course like this.
If it were a two semester course, perhaps I would!  Three things
come immediately to mind.

\begin{itemize}
\item LA for data science (c.f.~CS 6241)
  \begin{itemize}
  \item Non-negative matrix factorizations
  \item Tensors and tensor factorizations
  \item More on factorization-based methods in stats/ML
  \item The linear algebra of multivariate normals
  \item Connections to convex optimization: active sets, quadratic programming, etc
  \end{itemize}
\item Iterative methods (c.f.~CS 6220)
  \begin{itemize}
  \item More on multigrid and domain decomposition
  \item More on other ``data-sparse'' matrices
  \item More on elliptic PDEs, integral equations, etc
  \end{itemize}
\item Eigensolvers
  \begin{itemize}
  \item More on eigensolvers (especially iterative ones)
  \item Much more on perturbation theory and sensitivity analysis
  \item Matrix functions, and complex analysis connections
  \item Connections to control theory
  \item More on orthogonal polynomials
  \end{itemize}
\end{itemize}

But there is always more to learn.  If the course gave you a starting
point to thinking about other corners of linear algebra that you care
about for your research, then it was a success.

I enjoyed the class this semester.  I hope you did as well.

\end{document}
