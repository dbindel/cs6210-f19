\documentclass[12pt, leqno]{article} %% use to set typesize
\usepackage{fancyhdr}
\usepackage[sort&compress]{natbib}
\usepackage[letterpaper=true,colorlinks=true,linkcolor=black]{hyperref}

\input{commonm}

\newcommand{\hdr}[2]{
  \pagestyle{fancy}
  \lhead{Bindel, Fall 2019}
  \rhead{Matrix Computation}
  \fancyfoot{}
  \begin{center}
    {\large{\bf HW for #1}} \\ (due: #2)
  \end{center}
  \lstset{language=matlab,columns=flexible}
}


\begin{document}

\hdr{2019-09-27}

\section{Diagonally dominant matrices}

A matrix $A$ is {\em strictly (column) diagonally dominant} if
for each column $j$,
\[
  |a_{jj}| > \sum_{i \neq j} |a_{ij}|.
\]
If we write $A = D + F$ where $D$ is the diagonal and $F$ the
off-diagonal part, strict column diagonal dominance is equivalent
to the statement that
\[
  \|FD_{-1}\|_1 < 1.
\]
Note that we may factor $A$ as
\[
  A = (I+FD^{-1}) D
\]
with $D$ invertible because the diagonal elements are bounded below
by zero and $I+FD^{-1}$ invertible by a Neumann series bound.
Therefore $A$ is invertible if it is strictly column diagonally
dominant.

Strict diagonal dominance is a useful structural condition for
several reasons: it ensures nonsingularity, it guarantees convergence
of certain iterative methods (we will return to this later), and it
guarantees that $LU$ factorization can be done without pivoting.
In fact, Gaussian elimination without partial pivoting is guaranteed
not to even attempt pivoting!  To see this, note that the statement
is obvious for the first step: column diagonal dominance implies
that $a_{11}$ is the largest magnitude element in the first column.
What does the Schur complement look like after one step of Gaussian
elimination?  By a short computation, it turns out that the Schur
complement is again diagonally dominant (see GVL section 4.1.1).

Diagonally dominant matrices and symmetric positive definite matrices
are the two major classes of matrices for which unpivoted Gaussian
elimination is backward stable.

\section{Tridiagonal systems}

Consider a symmetric positive definite tridiagonal system
\[
  A = \begin{bmatrix}
  \alpha_1 & \beta_1 \\
  \beta_1 & \alpha_2 & \beta_2 \\
  & \beta_2 & \alpha_3 & \beta_3 \\
  & & \ddots & \ddots & \ddots \\
  & & & \beta_{n-2} & \alpha_{n-1} & \beta_{n-1} \\
  & & & & \beta_{n-1} & \alpha_n
  \end{bmatrix}
\]
If we do one step of Cholesky factorization, the first column of
multipliers is nonzero only in the first two entries,
while the Schur complement is
\[
  S = \begin{bmatrix}
  \alpha_2-\beta_1^2/\alpha_1 & \beta_2 \\
  \beta_2 & \alpha_3 & \beta_3 \\
  & \ddots & \ddots & \ddots \\
  & & \beta_{n-2} & \alpha_{n-1} & \beta_{n-1} \\
  & & & \beta_{n-1} & \alpha_n
  \end{bmatrix}.
\]
That is, at the first step, $L$ and $S$ retain {\em exactly the same
nonzero structure} as the original tridiagonal $A$.  Cholesky
factorization on a tridiagonal therefore runs in $O(n)$ time.

More generally, unpivoted {\em band elimination} retains the structure
of the $A$ matrix in the $LU$ factors: if $A$ has lower and upper
bandwidths $p$ and $q$, then $L$ and $U$ have lower and upper
bandwidths $p$ and $q$, respectively.  With pivoting, the upper
bandwidth of $L$ can go up to $p+q$, and there are at most
$p+1$ nonzeros per column of $L$.

LAPACK has specialized LU routines for SPD and general nonsymmetric
tridiagonal and banded matrices.

\section{Low-rank updates and bordered systems}

We have already discussed block elimination: given a system
\[
  \begin{bmatrix} A & B \\ C & D \end{bmatrix}
  \begin{bmatrix} x \\ y \end{bmatrix} =
  \begin{bmatrix} f \\ g \end{bmatrix}
\]
we can compute
\begin{align*}
  S &= D-C A^{-1} B \\
  S y &= g-C A^{-1} f \\
  A x &= f-By.
\end{align*}
If $A$ has some structure such that solves with $A$ are simple,
we may want to use this block solve structure rather than
forming and factoring the bordered system.

Let us now consider a closely-related problem: suppose we want
to solve
\[
  (A+UW^T) x = f.
\]
This looks like a problem, but what happens if we define an
intermediate variable $y = W^T x$?  If we put the original
equation in terms of $x$ and $y$, and we add the equation
relating $x$ and $y$, we get a bordered system
\[
  \begin{bmatrix} A & U \\ W^T & -I \end{bmatrix}
  \begin{bmatrix} x \\ y \end{bmatrix} =
  \begin{bmatrix} f \\ 0 \end{bmatrix}.
\]
Solving this bordered system by block elimination and
collapsing away all the intermediate variables gives us
\begin{align*}
  x &= A^{-1} \left( f - U y \right) \\
    &= A^{-1} \left( f - U S^{-1} (-W^T A^{-1} f) \right) \\
    &= \left( A^{-1} - A^{-1} U (I+W^T U V)^{-1} W^T A^{-1} \right) f.
\end{align*}
We can re-interpret this as the {\em Sherman-Morrison-Woodbury formula}:
\[
  (A+UW^T)^{-1} = A^{-1} - A^{-1} U (I+W^T A^{-1} U)^{-1} W^T A^{-1}.
\]
The bordered system formulation (together with iterative refinement
to clean up potential issues with instability) is quite useful as a
solver.  I prefer it to the presentation of the SMW formula as a
{\em fait accompli}.

\section{Vandermonde matrices}

The case of {\em Vandermonde matrices} is interesting for several
reasons:
\begin{itemize}
  \item They are highly structured.
  \item They are horribly conditioned.
  \item The ill-conditioned matrix appears as an intermediate
    in a problem that may be just fine.
\end{itemize}

A Vandermonde matrix is a matrix $V \in \bbR^{n \times n}$
whose entries are
\[
  v_{ij} = \xi_i^{j-1}.
\]
The matrix appears in polynomial interpolation.  The linear
system $Vc = f$ is equivalent to the conditions
\[
  p(\xi_i) = \sum_{j=1}^n c_j \xi_i^{j-1} = f_i.
\]
Assuming the $\xi$ are all distinct, this system is nonsingular.
However, the condition number grows {\em exponentially} as a
function of $n$.  Does this mean that the problem of polynomial
interpolation is horribly ill-conditioned?  Of course not!
The exponential ill-conditioning has to do with the expression
of the polynomial as a linear combination of monomials (the
so-called power basis).   But {\em we don't care} what the coefficient
vector $c$ will be; we just want a representation for the
interpolating polynomial $p$ that we can evaluate at points
other than the $\xi_i$.  If we represent $p$ in a different basis,
we often get a problem that is perfectly well-behaved.

\section{Circulant matrices}

We previously discussed fast {\em multiply} routines for circulant,
Toeplitz, and Hankel matrices.  The building block
to compute $y = Cx$ where $C$ is the circulant matrix with leading
column $c$ is to use the FFT to reduce the problem to diagonal form.
If $Z$ is the FFT matrix, we have
\[
  y = Cx = Z^{-1} \ddiag(\tilde{c}) Z x
\]
where $\tilde{c} = Zc$ is the FFT of $c$.  In code, we have
\begin{lstlisting}
y = ifft(fft(c) .* fft(x));
\end{lstlisting}
To solve a circulant system, we invert each of the linear operations
involved, i.e.
\[
  x = Z^{-1} \ddiag(\tilde{c})^{-1} Zy
\]
which we can implement in code as
\begin{lstlisting}
x = ifft(fft(y) ./ fft(c));
\end{lstlisting}
What if we wanted to solve a Toeplitz or a Hankel matrix?  There are
indeed fast Toeplitz and Hankel solvers, but they are much more subtle
than two Fourier transforms and a scaling.

\section{Other structures}

Are these the only structures that we can use for fast linear solves?
Of course they are not.  For any data sparse matrix, there is at least
the hope of a fast direct solver, whether it is based on Gaussian
elimination or some other approach.  But in many cases, it is not
as straightforward to come up with a fast linear solver as to come
up with a fast multiplication routine.

If you are faced with an unfamiliar structured matrix and want to
devise a fast solver, then, what is my best advice?  First, figure
out the name of the structure!  Then you can start searching to see
whether someone else has already devised a fast solver.  And if you
come up with a new idea, knowing the name of the structure you have
used increases the likelihood that your work may be re-used by
someone else. 

\end{document}
