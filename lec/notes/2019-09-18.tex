\documentclass[12pt, leqno]{article} %% use to set typesize
\usepackage{fancyhdr}
\usepackage[sort&compress]{natbib}
\usepackage[letterpaper=true,colorlinks=true,linkcolor=black]{hyperref}

\input{commonm}

\newcommand{\hdr}[2]{
  \pagestyle{fancy}
  \lhead{Bindel, Fall 2019}
  \rhead{Matrix Computation}
  \fancyfoot{}
  \begin{center}
    {\large{\bf HW for #1}} \\ (due: #2)
  \end{center}
  \lstset{language=matlab,columns=flexible}
}


\begin{document}

\hdr{2019-09-18}

\section{Introduction}

For the next few lectures, we will explore methods to solve linear
systems.  Our main tool will be the factorization $PA = LU$, where
$P$ is a permutation, $L$ is a unit lower triangular matrix, and $U$ is
an upper triangular matrix.  As we will see, the Gaussian elimination
algorithm learned in a first linear algebra class implicitly computes
this decomposition; but by thinking about the decomposition explicitly,
we can come up with other organizations for the computation.

We emphasize a few points up front:
\begin{itemize}
\item {\em Some matrices are singular.}  Errors in this part of the class
  often involve attempting to invert a matrix that has no inverse.
  A matrix does {\em not} have to be invertible to admit an LU factorization.
  We will also see more subtle problems from {\em almost} singular matrices.
\item {\em Some matrices are rectangular.}  In this part of the class,
  we will deal almost exclusively with square matrices; if a rectangular
  matrix shows up, we will try to be explicit about dimensions.  That
  said, LU factorization makes sense for rectangular matrices as well
  as for square matrices --- and it is sometimes useful.
\item {\em {\tt inv} is evil.}  The {\tt inv} command is one of the most
  abused commands in \matlab.  The \matlab\ backslash operator is the
  preferred way to solve a linear system absent other information:
\begin{lstlisting}
x = A \ b;  % Good
x = inv(A) * b;  % Evil
\end{lstlisting}
  Homework solutions that feature inappropriate explicit {\tt inv}
  commands {\em will} lose points.
\item {\em LU is not for linear solves alone.}  One can solve a variety
  of other interesting problems with an LU factorization.
\item {\em LU is not the only way to solve systems.}  Gaussian elimination
  and variants will be our default solver, but there are other solver
  methods that are appropriate for problems with more structure.
  We will touch on other methods throughout the class.
\end{itemize}

\section{Triangular solves}

Suppose that we have computed a factorization $PA = LU$.  How can we
use this to solve a linear system of the form $Ax = b$?  Permuting the
rows of $A$ and $b$, we have
\[
  PAx = LUx = Pb,
\]
and therefore
\[
  x = U^{-1} L^{-1} Pb.
\]
So we can reduce the problem of finding $x$ to two simpler problems:
\begin{enumerate}
\item
  Solve $Ly = Pb$
\item
  Solve $Ux = y$
\end{enumerate}
We assume the matrix $L$ is unit lower triangular (diagonal of all
ones + lower triangular), and $U$ is upper triangular, so we can solve
linear systems with $L$ and $U$ involving forward and backward
substitution.

As a concrete example, suppose
\[
  L = \begin{bmatrix}
        1 & 0 & 0 \\
        2 & 1 & 0 \\
        3 & 2 & 1
      \end{bmatrix}, \quad
  d = \begin{bmatrix} 1 \\ 1 \\ 3 \end{bmatrix}
\]
To solve a linear system of the form $Ly = d$, we process each row in
turn to find the value of the corresponding entry of $y$:
\begin{enumerate}
\item Row 1:
  $y_1 = d_1$
\item Row 2: $2y_1 + y_2 = d_2$, or $y_2 = d_2 - 2y_1$
\item Row 3: $3y_1 + 2 y_2 + y_3 = d_3$, or $y_3 = d_3 - 3y_1 - 2y_2$
\end{enumerate}
More generally, the {\em forward substitution} algorithm for solving
unit lower triangular linear systems $Ly = d$ looks like
\begin{lstlisting}
  y = d;
  for i=2:n
    y(i) = d(i)-L(1:i-1)*y(1:i-1)
  end
\end{lstlisting}
Similarly, there is a {\em backward substitution} algorithm for
solving upper triangular linear systems $Ux = d$
\begin{lstlisting}
  x(n) = d(n)/U(n,n);
  for i=n-1:-1:1
    x(i) = ( d(i)-U(i+1:n)*x(i+1:n) )/U(i,i)
  end
\end{lstlisting}
Each of these algorithms takes $O(n^2)$ time.

%% \section{Triangular matrix graphs and groups}

%% Before moving on from triangular matrices, it is worth pointing
%% out two useful structural facts.

%% \subsection{Triangular matrices and DAGs}

%% Triangular matrices are associated with {\em directed acyclic graphs}
%% (DAGs), and this is part of what makes them useful for solving linear
%% systems: forward and back-substitution are both instances of processing
%% the graph according to a topological ordering by variable dependencies.

%% In \matlab, a {\em psychologically triangular} matrix is a matrix whose
%% rows can be permuted to obtain a lower (or upper) triangular matrix.  A
%% \matlab\ solve involving a psychologically triangular matrix is still
%% $O(n^2)$, since \matlab\ checks in advance for this structure.  When the
%% {\tt lu} function in \matlab\ is called without explicitly returning a
%% permutation, the $L$ matrix it returns is psychologically lower
%% triangular.

%% In principle, \matlab\ could use a topological sort to solve a
%% matrix that could be transformed to triangularity by row {\em and column}
%% permutations in $O(n^2)$ time.  In practice, it does not!

%% \subsection{Triangular matrices and groups}

%% % Psychologically triangular matrices and DAGs
%% % Matrix groups

%% The (square) unit lower triangular matrices (lower triangular matrices with
%% ones on the main diagonal) have several interesting properties:
%% \begin{itemize}
%% \item $I$ is unit lower triangular
%% \item Products of unit lower triangular matrices are unit lower triangular
%% \item Every unit lower triangular matrix has a unit lower triangular
%%   inverse
%% \end{itemize}
%% That is, the unit lower triangular matrices form an {\em algebraic group}
%% within the set of square matrices.  If you don't know what an algebraic
%% group is, that is fine --- but it is worth noting that the unit triangular
%% matrices are closed under inversion and multiplication.

%% The invertible lower triangular matrices are also an algebraic group,
%% as are the unit upper triangular matrices and the invertible upper
%% triangular matrices.

\section{Gaussian elimination by example}

Let's start our discussion of $LU$ factorization by working through
these ideas with a concrete example:
\[
  A =
  \begin{bmatrix}
    1 & 4 & 7 \\
    2 & 5 & 8 \\
    3 & 6 & 10
  \end{bmatrix}.
\]
To eliminate the subdiagonal entries $a_{21}$ and $a_{31}$, we
subtract twice the first row from the second row, and thrice the
first row from the third row:
\[
  A^{(1)} =
  \begin{bmatrix}
    1 & 4 & 7 \\
    2 & 5 & 8 \\
    3 & 6 & 10
  \end{bmatrix} -
  \begin{bmatrix}
    0 \cdot 1 & 0 \cdot 4 & 0 \cdot 7 \\
    2 \cdot 1 & 2 \cdot 4 & 2 \cdot 7 \\
    3 \cdot 1 & 3 \cdot 4 & 3 \cdot 7
  \end{bmatrix}
  =
  \begin{bmatrix}
    1 &  4 &  7 \\
    0 & -3 & -6 \\
    0 & -6 & -11
  \end{bmatrix}.
\]
That is, the step comes from a rank-1 update to the matrix:
\[
  A^{(1)} =
  A -
  \begin{bmatrix} 0 \\ 2 \\ 3 \end{bmatrix}
  \begin{bmatrix} 1 & 4 & 7 \end{bmatrix}.
\]
Another way to think of this step is as a linear transformation
$A^{(1)} = M_1 A$, where the rows of $M_1$ describe the multiples
of rows of the original matrix that go into rows of the updated matrix:
\[
  M_1 = \begin{bmatrix} 1 & 0 & 0 \\ -2 & 1 & 0 \\ -3 & 0 & 1 \end{bmatrix}
      = I - \begin{bmatrix} 0 \\ 2 \\ 3 \end{bmatrix}
            \begin{bmatrix} 1 & 0 & 0 \end{bmatrix}
      = I - \tau_1 e_1^T.
\]
Similarly, in the second step of the algorithm, we subtract twice the second
row from the third row:
\[
  \begin{bmatrix}
    1 &  4 &  7 \\
    0 & -3 & -6 \\
    0 &  0 &  1
  \end{bmatrix} =
  \begin{bmatrix}
    1 & 0 & 0 \\
    0 & 1 & 0 \\
    0 & -2 & 1
  \end{bmatrix}
  \begin{bmatrix}
    1 &  4 &  7 \\
    0 & -3 & -6 \\
    0 & -6 & -11
  \end{bmatrix} =
  \left( I - \begin{bmatrix} 0 \\ 0 \\ 2 \end{bmatrix}
             \begin{bmatrix} 0 & 1 & 0 \end{bmatrix} \right) A^{(1)}.
\]
More compactly: $U = (I-\tau_2 e_2^T) A^{(1)}$.

Putting everything together, we have computed
\[
  U = (I-\tau_2 e_2^T) (I-\tau_1 e_1^T) A.
\]
Therefore,
\[
  A = (I-\tau_1 e_1^T)^{-1} (I-\tau_2 e_2^T)^{-1} U = LU.
\]
Now, note that
\[
  (I-\tau_1 e_1^T) (I + \tau_1 e_1^T) =
  I - \tau_1 e_1^T + \tau_1 e_1^T - \tau_1 e_1^T \tau_1 e_1^T = I,
\]
since $e_1^T \tau_1$ (the first entry of $\tau_1$) is zero.  Therefore,
\[
  (I-\tau_1 e_1^T)^{-1} = (I+\tau_1 e_1^T)
\]
Similarly,
\[
  (I-\tau_2 e_2^T)^{-1} = (I+\tau_2 e_2^T)
\]
Thus,
\[
  L = (I+\tau_1 e_1^T)(I + \tau_2 e_2^T).
\]
Now, note that because $\tau_2$ is only nonzero in the third element,
$e_1^T \tau_2 = 0$; thus,
\begin{align*}
  L &= (I+\tau_1 e_1^T)(I + \tau_2 e_2^T) \\
    &= (I + \tau_1 e_1^T + \tau_2 e_2^T + \tau_1 (e_1^T \tau_2) e_2^T \\
    &= I + \tau_1 e_1^T + \tau_2 e_2^T \\
    &= \begin{bmatrix} 1 & 0 & 0 \\ 0 & 1 & 0 \\ 0 & 0 & 1 \end{bmatrix} +
       \begin{bmatrix} 0 & 0 & 0 \\ 2 & 0 & 0 \\ 3 & 0 & 0 \end{bmatrix} +
       \begin{bmatrix} 0 & 0 & 0 \\ 0 & 0 & 0 \\ 0 & 2 & 0 \end{bmatrix}
     = \begin{bmatrix} 1 & 0 & 0 \\ 2 & 1 & 0 \\ 3 & 2 & 1 \end{bmatrix}.
\end{align*}

The final factorization is
\[
A =
\begin{bmatrix} 1 & 4 & 7 \\ 2 & 5 & 8 \\ 3 & 6 & 10 \end{bmatrix} =
\begin{bmatrix} 1 & 0 & 0 \\ 2 & 1 & 0 \\ 3 & 2 & 1 \end{bmatrix}
\begin{bmatrix} 1 & 4 & 7 \\ 0 & -3 & -6 \\ 0 & 0 & 1 \end{bmatrix} = LU.
\]
Note that the subdiagonal elements of $L$ are easy to read off: for $j
> i$, $l_{ij}$ is the multiple of row $j$ that we subtract from row
$i$ during elimination.  This means that it is easy to read off the
subdiagonal entries of $L$ during the elimination process.

\section{Aside: Gauss transformations and shearing}

In the previous section, we observed that we can reduce a matrix
to upper triangularity by repeated multiplication by Gauss
transformations
\[
  I-\tau_k e_k^T
\]
where $\tau_k$ is nonzero only in elements below the main diagonal.
It is worth thinking about what this means geometrically: a Gauss
transformation is a {\em shear} transformation affecting a particular
coordinate direction.  If we think about the columns of a matrix $A$
as representing the edges of a paralellipiped in $\bbR^n$, then
Gaussian elimination can be interpreted as the process of applying
shear transformations to move the first vector into the $x$ direction,
the second into the $xy$-plane, the third into the $xyz$ space, and
so forth.  Because shear transformations do not change volume, the
transformed parallelipiped has the same volume as the original one.
But because of the axis alignment in the transformed parallelipiped,
we can compute the volume via the generalization of the
usual ``base $\times$ height'' computation.

Algebraically, what have we just observed?  If $A = LU$ where $L$
is unit lower triangular and $U$ is upper triangular, then
\[
  \det(A) = \det(L) \det(U) = \det(U) = \prod_{k=1}^n u_{kk}.
\]
That is, we can use the LU factorization to compute determinants
or volumes as well as to solve linear systems.  I prefer to start
from the perspective of volume-preserving shear transformations,
though, as I consider this a very natural explanation for why
determinants have anything to do with volume.  Indeed, I pretty much
took the volume characterization of determinants as a matter of faith
rather than true understanding, up to the point where I really
understood the connection to various matrix factorizations.  We will
see this connection again when we talk about QR factorization and
least squares.

\section{Basic LU factorization}

Let's generalize our previous algorithm and write a simple code for
$LU$ factorization.  We will leave the issue of pivoting to a later
discussion.  We'll start with a purely loop-based implementation:
\begin{lstlisting}
%
% Overwrites A with an upper triangular factor U, keeping track of
% multipliers in the matrix L.
%
function [L,A] = mylu(A)

  n = length(A);
  L = eye(n);
  for j=1:n-1
    for i=j+1:n

      % Figure out multiple of row j to subtract from row i
      L(i,j) = A(i,j)/A(j,j);

      % Subtract off the appropriate multiple
      A(i,j) = 0
      for k=j+1:n
        A(i,k) = A(i,k) - L(i,j)*A(j,k);
      end
    end
  end
\end{lstlisting}
Note that we can write the two innermost loops more concisely by
thinking of them in terms of applying a {\em Gauss
  transformation} $M_j = I - \tau_j e_j^T$, where $\tau_j$ is the
vector of multipliers that appear when eliminating in column $j$:
\begin{lstlisting}
%
% Overwrites A with an upper triangular factor U, keeping track of
% multipliers in the matrix L.
%
function [L,A] = mylu(A)

  n = length(A);
  L = eye(n);
  for j=1:n-1

    % Form vector of multipliers
    L(j+1:n,j) = A(j+1:n,j)/A(j,j);

    % Apply Gauss transformation
    A(j+1:n,j) = 0;
    A(j+1:n,j+1:n) = A(j+1:n,j+1:n)-L(j+1:n,j)*A(j,j+1:n);

  end
\end{lstlisting}

\end{document}
