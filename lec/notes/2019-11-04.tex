\documentclass[12pt, leqno]{article} %% use to set typesize
\usepackage{fancyhdr}
\usepackage[sort&compress]{natbib}
\usepackage[letterpaper=true,colorlinks=true,linkcolor=black]{hyperref}

\input{commonm}

\newcommand{\hdr}[2]{
  \pagestyle{fancy}
  \lhead{Bindel, Fall 2019}
  \rhead{Matrix Computation}
  \fancyfoot{}
  \begin{center}
    {\large{\bf HW for #1}} \\ (due: #2)
  \end{center}
  \lstset{language=matlab,columns=flexible}
}


\usepackage{amsthm}
\newtheorem{theorem}{Theorem}

\begin{document}

\hdr{2019-11-04}

\section{Minimax and interlacing}

The Rayleigh quotient is a building block for a great deal of theory.
One step beyond the basic characterization of eigenvalues as stationary
points of a Rayleigh quotient, we have
the {\em Courant-Fischer minimax theorem}:
\begin{theorem}
  If $\lambda_1 \geq \lambda_2 \geq \ldots \geq \lambda_n$, then
  we can characterize the eigenvalues via optimizations over
  subspaces $\calV$:
  \[
    \lambda_k
      = \max_{\dim \calV = k} \left( \min_{0 \neq v \in \calV} \rho_A(v) \right) \\
      = \min_{\dim \calV = n-k+1} \left( \max_{0 \neq v \in \calV} \rho_A(v) \right).
  \]
\end{theorem}

\begin{proof}
  Write $A = U \Lambda U^*$ where $U$ is a unitary matrix of eigenvectors.
  If $v$ is a unit vector, so is $x = U^* v$, and we have
  \[
    \rho_A(v) = x^* \Lambda x = \sum_{j=1}^n \lambda_j |x_j|^2,
  \]
  i.e. $\rho_A(v)$ is a weighted average of the eigenvalues
  of $A$.  If $\calV$ is a $k$-dimensional subspace, then we can find
  a unit vector $v \in \calV$ that satisfies the $k-1$ constraints
  $(U^* v)_j = 0$ for $j = 1$ through $k-1$ (i.e. $v$ is orthogonal to
  the invariant subspace associated with the first $k-1$
  eigenvectors).  For this $v$, $\rho_A(v)$ is a weighted average of
  $\lambda_k, \lambda_{k+1}, \ldots, \lambda_n$, so $\rho_A(v) \leq
  \lambda_k$.  Therefore,
  \[
      \max_{\dim \calV = k} \left( \min_{0 \neq v \in \calV} \rho_A(v) \right)
      \leq \lambda_k.
  \]
  Now, if $\calV$ is the range space of the first $k$ columns of $U$,
  then for any $v \in \calV$ we have that $\rho_A(v)$ is a weighted
  average of the first $k$ eigenvalues, which attains the minimal value
  $\lambda_k$ when we choose $v = u_k$.
\end{proof}

One piece of the minimax theorem is that given any $k$-dimensional subspace
$\calV$, the smallest value of the Rayleigh quotient over that
subspace is a {\em lower} bound on $\lambda_k$ and an {\em upper} bound
on $\lambda_{n-k+1}$.  Taking this one step further, we have the {\em Cauchy
interlace theorem}, which relates the eigenvalues of a block Rayleigh quotient
to the eigenvalues of the corresponding matrix.
\begin{theorem}
  Suppose $A$ is real symmetric (or Hermitian), and let $V$ be a
  matrix with $m$ orthonormal columns.  Then the eigenvalues of
  $W^* A W$ interlace the eigenvalues of $A$; that is, if $A$
  has eigenvalues $\alpha_1 \geq \alpha_2 \geq \ldots \geq \alpha_n$
  and $W^* A W$ has eigenvalues $\beta_j$, then
  \[
    \beta_j \in [\alpha_{n-m+j}, \alpha_j].
  \]
\end{theorem}

\begin{proof}
  Suppose $A \in \bbC^{n \times n}$ and $L \in \bbC^{m \times m}$.  The matrix
  $W$ maps $\bbC^{m}$ to $\bbC^{n}$, so for each $k$-dimensional subspace
  $\calV \subseteq \bbC^{m}$ there is a corresponding
  $k$-dimensional subspace of $W\calV \subseteq \bbC^{n}$.  Thus,
  \begin{align*}
  \beta_j &=
    \max_{\dim \calV = k} \left( \min_{0 \neq v \in \calV} \rho_L(v) \right)
  = \max_{\dim \calV = k} \left( \min_{0 \neq v \in W\calV} \rho_A(v) \right)
  \leq \alpha_k
  \end{align*}
  and similarly
  \begin{align*}
  \beta_j &=
    \min_{\dim \calV = m-k+1} \left( \max_{0 \neq v \in \calV} \rho_L(v) \right)
  =
    \min_{\dim \calV = m-k+1} \left( \max_{0 \neq v \in W\calV} \rho_A(v) \right) \\
  &=
    \min_{\dim \calV = n-(k+(n-m))+1} \left( \max_{0 \neq v \in W\calV} \rho_A(v) \right)
   \geq \alpha_{n-m+k}
  \end{align*}
\end{proof}

Another application of the minimax theorem is due to Weyl:
if we write $\lambda_k(A)$ for the $k$th largest eigenvalue of
a symmetric $A$, then for any symmetric $A$ and $E$,
\[
  |\lambda_k(A+E)-\lambda_k(A)| \leq \|E\|_2.
\]
A related theorem is the Wielandt-Hoffman theorem:
\[
  \sum_{i=1}^n (\lambda_i(A+E)-\lambda_i(A))^2 \leq \|E\|_F^2.
\]
Both these theorems provide strong information about the spectrum
relative to what we have in the nonsymmetric case (e.g. from
Bauer-Fike).  Not only do we know that each eigenvalue of $A+E$ is
close to {\em some} eigenvalue of $A$, but we know that we can put the
eigenvalues of $A$ and $A+E$ into one-to-one correspondence.  So for
the eigenvalues in the symmetric case, small backward error implies
small forward error!

As an aside, note that if $\hat{v}$ is an approximate eigenvector and
$\hat{\lambda} = \rho_A(\hat{v})$ for a symmetric $A$, then we can
find an explicit form for a backward error $E$ such that
\[
  (A+E)\hat{v} = \hat{v}\hat{\lambda}.
\]
by evaluate the residual $r = Av-v\lambda$ and writing $E = rv^* + vr^*$.
So in the symmetric case, a small residual implies that we are near an
eigenvalue.  On the other hand, it says little about the corresponding
eigenvector, which may still be very sensitive to perturbations if
it is associated with an eigenvalue that is close to other eigenvalues.

\section{Sensitivity of invariant subspaces}

The eigenvalues of a symmetric matrix are perfectly conditioned.  What
of the eigenvectors (or, more generally, the invariant subspaces)?
Here the picture is more complex, and involves {\em spectral gaps}.
Suppose $u$ is an eigenvector of $A$ associated with eigenvalue $\mu$,
and the nearest other eigenvalue is at least $\gamma$ apart.  Then
there is a perturbation $E$ with $\|E\|_2 = \gamma/2$ for which the
eigenvalue at $\mu$ and the nearest eigenvalue coalesce.

A more refined picture is given by Davis and Kahan and covered in many
textbooks since (I recommend those of Parlett and of Stewart).  Let
$AU = U\Lambda$ and $\hat{A} \hat{U} = \hat{U} \hat{\Lambda}$,
and define $R = \|\hat{A} U-U \Lambda\|$.  Then
\[
  \|\sin \Theta(U,\hat(U))\|_F \leq \frac{\|R\|_F}{\delta}
\]
where $\delta$ is the gap between the eigenvalues in $\Lambda$
and the rest of the spectrum.  If we enforce a gap between an interval
containing the eigenvalues in $\Lambda$ and the rest of the spectrum,
we can change all the Frobenius norms into 2-norms (or any other
unitarily invariant norm).  The matrix $\sin \Theta(U,\hat{U})$ is
the matrix of sines of the {\em canonical angles} between $U$ and $\hat{U}$;
if both bases are normalized, the cosines of these canonical angles are
the singular values of $U^* \hat{U}$.

The punchline for this is that an eigenvector or invariant subspace
for eigenvalues separated by a large spectral gap from everything
else in the specturm is nicely stable.  But if the spectral gap is small,
the vectors may spin like crazy under perturbations.

\section{Sylvester's inertia theorem}

The inertia $\nu(A)$ is a triple consisting of the number of positive,
negative, and zero eigenvalues of $A$.  {\em Sylvester's inertia
  theorem} says that inertia is preserved under nonsingular {\em
  congruence} transformations, i.e. transformations of the form
\[
  M = V^* A V
\]
where $V$ is nonsingular (but not necessarily unitary).

Congruence transformations are significant because they are the natural
transformations for {\em quadratic forms} defined by symmetric matrices;
and the invariance of inertia under congruence says something about the
invariance of the shape of a quadratic form under a change of basis.
For example, if $A$ is a positive (negative) definite matrix, then the
quadratic form
\[
  \phi(x) = x^* A x
\]
defines a concave (convex) bowl; and $\phi(Vx) = x^* (V^* A V) x$ has
the same shape.

As with almost anything else related to the symmetric eigenvalue
problem, the minimax characterization is the key to proving
Sylvester's inertia theorem.  The key observation is that if
$M = V^* A V$ and $A$ has $k$ positive eigenvalues, then the minimax theorem
gives us a $k$-dimensional subspace $\calW_+$ on which $A$ is positive
definite (i.e. if $W$ is a basis, then $z^* (W^* A W) z > 0$ for any
nonzero $z$).  The matrix $M$ also has a $k$-dimensional space on
which it is positive definite, namely $V^{-1} \calW$.  Similarly, $M$
and $A$ both have $(n-k)$-dimensional spaces on which they are
negative semidefinite.  So the number of positive eigenvalues of $M$
is $k$, just as the number of positive eigenvalues of $A$ is $k$.

\end{document}
