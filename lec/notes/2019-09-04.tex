\documentclass[12pt, leqno]{article} %% use to set typesize
\usepackage{fancyhdr}
\usepackage[sort&compress]{natbib}
\usepackage[letterpaper=true,colorlinks=true,linkcolor=black]{hyperref}

\input{commonm}

\newcommand{\hdr}[2]{
  \pagestyle{fancy}
  \lhead{Bindel, Fall 2019}
  \rhead{Matrix Computation}
  \fancyfoot{}
  \begin{center}
    {\large{\bf HW for #1}} \\ (due: #2)
  \end{center}
  \lstset{language=matlab,columns=flexible}
}


\begin{document}

\hdr{2019-09-04}


\section{Logistics}

Office hours for the semester have been finalized; these are
\begin{itemize}
\item JPR: Tues 1:30-2:30 in Rhodes 503
\item DSB: Weds 1:30-2:30 in Gates 425
\item JPR: Weds 2:30-3:30 in Rhodes 503
\item DSB: Fri 10-11 in Gates 425
\end{itemize}
I am also available by scheduled appointment.

\section{Matrix shapes and structures}

In linear algebra, we talk about different matrix structures.
For example:
\begin{itemize}
\item $A \in \bbR^{n \times n}$ is {\em nonsingular} if the inverse
  exists; otherwise it is {\em singular}.
\item $Q \in \bbR^{n \times n}$ is {\em orthogonal} if $Q^T Q = I$.
\item $A \in \bbR^{n \times n}$ is {\em symmetric} if $A = A^T$.
\item $S \in \bbR^{n \times n}$ is {\em skew-symmetric} if $S = -S^T$.
\item $L \in \bbR^{n \times m}$ is {\em low rank} if $L = UV^T$
  for $U \in \bbR^{n \times k}$ and $V \in \bbR^{m \times k}$ where
  $k \ll \min(m,n)$.
\end{itemize}
These are properties of an underlying linear map or quadratic form; if
we write a different matrix associated with an (appropriately
restricted) change of basis, it will also have the same properties.

In matrix computations, we also talk about the {\em shape} (nonzero
structure) of a matrix.  For example:
\begin{itemize}
\item $D$ is {\em diagonal} if $d_{ij} = 0$ for $i \neq j$.
\item $T$ is {\em tridiagonal} if $t_{ij} = 0$ for $i \not \in \{j-1,
  j, j+1\}$.
\item $U$ is {\em upper triangular} if $u_{ij} = 0$ for $i > j$
  and {\em strictly upper triangular} if $u_{ij} = 0$ for $i \geq j$
  (lower triangular and strictly lower triangular are similarly
  defined).
\item $H$ is {\em upper Hessenberg} if $h_{ij} = 0$ for $i > j+1$.
\item $B$ is {\em banded} if $b_{ij} = 0$ for $|i-j| > \beta$.
\item $S$ is {\em sparse} if most of the entries are zero.  The
  position of the nonzero entries in the matrix is called the
  {\em sparsity structure}.
\end{itemize}
We often represent the shape of a matrix by marking where the nonzero
elements are (usually leaving empty space for the zero elements); for
example:
\begin{align*}
  \mbox{Diagonal} &
  \begin{bmatrix}
    \times & & & & \\
    & \times & & & \\
    & & \times & & \\
    & & & \times & \\
    & & & & \times
  \end{bmatrix} &
  \mbox{Tridiagonal} &
  \begin{bmatrix}
    \times & \times & & & \\
    \times & \times & \times & & \\
    & \times & \times & \times & \\
    & & \times & \times & \times \\
    & & & \times & \times
  \end{bmatrix} \\
  \mbox{Triangular} &
  \begin{bmatrix}
    \times & \times & \times & \times & \times \\
    & \times & \times & \times & \times \\
    & & \times & \times & \times \\
    & & & \times & \times \\
    & & & & \times
  \end{bmatrix} &
  \mbox{Hessenberg} &
  \begin{bmatrix}
    \times & \times & \times & \times & \times \\
    \times & \times & \times & \times & \times \\
    & \times & \times & \times & \times \\
    & & \times & \times & \times \\
    & & & \times & \times
  \end{bmatrix} \\
\end{align*}
We also sometimes talk about the {\em graph} of a (square) matrix
$A \in \bbR^{n \times n}$: if we assign a node to each index
$\{1, \ldots, n\}$, an edge $(i,j)$ in the graph corresponds
to $a_{ij} \neq 0$.  There is a close connection between certain
classes of graph algorithms and algorithms for factoring sparse
matrices or working with different matrix shapes.  For example,
the matrix $A$ can be permuted so that $P A P^T$ is upper triangular
iff the associated directed graph is acyclic.

The shape of a matrix (or graph of a matrix) is not intrinsically
associated with a more abstract linear algebra concept; apart from
permutations, sometimes, almost any change of basis will completely
destroy the shape.

\section{Sparse matrices}

We say a matrix is {\em sparse} if the vast majority of the entries are
zero.  Because we only need to explicitly keep track of the nonzero
elements, sparse matrices require less than $O(n^2)$ storage, and we
can perform many operations more cheaply with sparse matrices than with
dense matrices.  In general, the cost to store a sparse matrix, and to
multiply a sparse matrix by a vector, is $O(\nnz(A))$, where $\nnz(A)$
is the {\em number of nonzeros} in $A$.

Two specific classes of sparse matrices are such ubiquitous building
blocks that it is worth pulling them out for special attention.
These are diagonal matrices and permutation matrices.  Many linear
algebra libraries also have support for {\em banded} matrices
(and sometimes for generalizations such as {\em skyline} matrices).
\matlab\ also provides explicit support for general sparse matrices
in which the nonzeros can appear in any position.

\subsection{Diagonal matrices}

A diagonal matrix is zero except for the entries on the diagonal.
We often associate a diagonal matrix with the vector of these entries,
and we will adopt in class the notational convention used in \matlab:
the operator $\ddiag$ maps a vector to the corresponding diagonal matrix,
and maps a matrix to the vector of diagonal entries.  For example,
for the vector and matrix
\[
  d = \begin{bmatrix} d_1 \\ d_2 \\ d_3 \end{bmatrix} \\
  D = \begin{bmatrix} d_1 & & \\ & d_2 & \\ & & d_3 \end{bmatrix}
\]
we would write $D = \ddiag(d)$ and $d = \ddiag(D)$.

The \matlab\ routine {\tt diag} forms a dense representation of a
diagonal matrix.  Next to {\tt inv}, it is one of the \matlab\ commands
that is most commonly poorly used.  The primary {\em good} use of
{\tt diag} is as the first term in a sum that builds a more complicated matrix.
But {\em multiplication} by a diagonal matrix should never go through the
{\tt diag} routine.  To multiple a diagonal matrix by a vector, the preferred
idiom is to use the elementwise multiplication operation, i.e.
\begin{lstlisting}
  y = diag(d) * x;  % Bad -- O(n^2) time and intermediate storage
  y = d .* x;  % Good -- equivalent, but O(n) time and space
\end{lstlisting}
To multiply a diagonal matrix by a vector in \matlab, use the {\tt bsxfun}
command, e.g.
\begin{lstlisting}
  B = diag(d) * A;  % Bad -- left scaling in O(n^3) time
  C = A * diag(d);  % Bad -- right scaling in O(n^3) time
  B = bsxfun(@times, d, A);  % Good -- O(n^2) time
  C = bsxfun(@times, A, d.');  % Ditto
\end{lstlisting}

\subsection{Permutations}

A permutation matrix is a 0-1 matrix in which one appears exactly
once in each row and column.  We typically use $P$ or $\Pi$ to
denote permutation matrices; if there are two permutations in a
single expression, we might use $P$ and $Q$.

A permutation matrix is so named because it permutes the entries
of a vector.  As with diagonal matrices, it is usually best to
work with permutations implicitly in computational practice.
For any given permutation vector $P$, we can define an associated
mapping vector $p$ such that $p(i) = j$ iff $P_{ij} = 1$.  We can
then apply the permutation to a vector or matrix using \matlab's
indexing operations:
\begin{lstlisting}
  B = P*A;    % Straightforward, but slow if P is a dense rep'n
  C = A*P';
  B = A(p,:); % Better
  C = A(:,p);
\end{lstlisting}
To apply a transpose permutation, we would usually use the permuted
indexing on the destination rather than the source:
\begin{lstlisting}
  y = P'*x;  % Implies that P*y = x
  y(p) = x;  % Apply the transposed permutation via indexing
\end{lstlisting}

\subsection{Narrowly banded matrices}

If a matrix $A$ has zero entries outside a narrow band near the
diagonal, we say that $A$ is a {\em banded} matrix.  More precisely,
if $a_{ij} = 0$ for $j < i-k_1$ or $j > i+k_2$, we say that $A$
has {\em lower bandwidth} $k_1$ and {\em upper bandwidth} $k_2$.
The most common narrowly-banded matrices in matrix computations
(other than diagonal matrices) are {\em tridiagonal} matrices in
which $k_1 = k_2 = 1$.

In the conventional storage layout for band matrices (used by LAPACK)
the nonzero entries for a band matrix $A$ are stored in a packed
storage matrix $B$ such that each column of $B$ corresponds to a column
of $A$ and each row of $B$ corresponds to a nonzero (off-)diagonal of $A$.
For example,
\[
  \begin{bmatrix}
    a_{11} & a_{12} \\
    a_{21} & a_{22} & a_{23} \\
    a_{31} & a_{32} & a_{33} & a_{34} \\
    & a_{42} & a_{43} & a_{44} & a_{45} \\
    & & a_{53} & a_{54} & a_{55}
  \end{bmatrix} \mapsto
  \begin{bmatrix}
    * & a_{12} & a_{23} & a_{34} & a_{45} \\
    a_{11} & a_{22} & a_{33} & a_{44} & a_{55} \\
    a_{21} & a_{32} & a_{43} & a_{54} & * \\
    a_{31} & a_{42} & a_{53} & * & *
  \end{bmatrix}
\]
\matlab\ does not provide easy specialized support for band matrices
(though it is possible to access the band matrix routines if you
are tricky).  Instead, the simplest way to work with narrowly banded
matrices in \matlab\ is to use a general sparse representation.

\subsection{General sparse matrices}

For diagonal and band matrices, we are able to store nonzero matrix
entries explicitly, but (as with the dense matrix format) the locations
of those nonzero entries in the matrix are implicit.  For permutation
matrices, the values of the nonzero entries are implicit (they are
always one), but we must store their positions explicitly.  In a
{\em general} sparse matrix format, we store both the positions and
the values of nonzero entries explicitly.

For input and output, \matlab\ uses a {\em coordinate} format for
sparse matrices consisting of three parallel arrays ({\tt i}, {\tt j},
and {\tt aij}).  Each entry in the parallel arrays represents a
nonzero in the matrix with value {\tt aij(k)} at row {\tt i(k)} and
column {\tt j(k)}.  For input, repeated entries with the same row and
column are allowed; in this case, all the entries for a given location
are summed together in the final matrix.  For example,
\begin{lstlisting}
  i = [1, 2, 3, 4, 4]; % Row indices
  j = [2, 3, 4, 5, 5]; % Col indices
  v = [5, 8, 13, 21, 34]; % Entry values/contributions
  A = sparse(i,j,v,5,5); % 5-by-5 sparse matrix
  full(A) % Convert to dense format and display

  % Output:
  % ans =
  %
  %   0    5    0    0    0
  %   0    0    8    0    0
  %   0    0    0   13    0
  %   0    0    0    0   55
  %   0    0    0    0    0
\end{lstlisting}
This functionality is useful
in some applications (e.g.~for assembling finite element matrices).

Internally, \matlab\ uses a {\em compressed sparse column} format
for sparse matrices.  In this format, the row position and value for
each nonzero are stored in parallel arrays, in column-major
order (i.e.~all the elements of column $k$ appear before elements of
column $k+1$).  The column positions are not stored explicitly for
every element; instead, a {\em pointer array} indicates the offset
in the row and entry arrays of the start of the data for each column;
a pointer array entry at position $n+1$ indicates the total number
of nonzeros in the data structure.

The compressed sparse column format has some features that may not
be obvious at first:
\begin{itemize}
\item
  For very sparse matrices, multiplying a sparse format matrix by a
  vector is much faster than multiplying a dense format matrix by a
  vector --- but this is not true if a significant fraction of the
  matrix is nonzeros.  The tradeoff depends on the matrix size and
  machine details, but sparse matvecs will often have the same speed as ---
  or even be slower than --- dense matvecs when the sparsity is above a
  few percent.
\item
  Adding contributions into a sparse matrix is relatively slow,
  as each sum requires recomputing the sparse indexing data structure
  and reallocating memory.  To build up a sparse matrix as the sum of
  many components, it is usually best to use the coordinate form first.
\end{itemize}
In general, though, the sparse matrix format has a great deal to
recommend it for genuinely sparse matrices.  \matlab\ uses the sparse
matrix format not only for general sparse matrices, but also for the
special case of banded matrices.

\section{Data-sparse matrices}

A {\em sparse} matrix has mostly zero entries; this lets us design
compact storage formats with space proportional to the number of nonzeros,
and fast matrix-vector multiplication with time proportional to the
number of nonzeros.  A {\em data-sparse} matrix can be described with
far fewer than $n^2$ parameters, even if it is not sparse.  Such matrices
usually also admit compact storage schemes and fast matrix-vector products.
This is significant because many of the iterative algorithms we
describe later in the semester do not require any particular representation
of the matrix; they only require that we be able to multiply by a vector
quickly.

The study of various data sparse representations has blossomed into a
major field of study within matrix computations; in this section we give
a taste of a few of the most common types of data sparsity.  We will
see several of these structures in model problems used over the course
of the class.

\subsection{(Nearly) low-rank matrices}

The simplest and most common data-sparse matrices are {\em low-rank}
matrices.  If $A \in \bbR^{m \times n}$ has rank $k$, it can be written
in outer product form as
\[
  A = UW^T, \quad, U \in \bbR^{m \times k}, W \in \bbR^{n \times k}.
\]
This factored form has a storage cost of $(n+m) k$, a significant savings
over the $mn$ cost of the dense representation in the case $k \ll \max(m,n)$.
To multiply a low-rank matrix by a vector fast, we need only to use
associativity of matrix operations
\begin{lstlisting}
  y = (U*V')*x;  % O(mn) storage, O(mnk) flops
  y = U*(V'*x);  % O((m+n) k) storage and flops
\end{lstlisting}

\subsection{Circulant, Toeplitz, and Hankel structure}

A {\em Toeplitz} matrix is a matrix in which each (off)-diagonal is
constant, e.g.
\[
  A =
  \begin{bmatrix}
    a_0    & a_1    & a_2    & a_3 \\
    a_{-1} & a_0    & a_1    & a_2 \\
    a_{-2} & a_{-1} & a_0    & a_1 \\
    a_{-3} & a_{-2} & a_{-1} & a_0
  \end{bmatrix}.
\]
Toeplitz matrices play a central role in the theory of constant-coefficient
finite difference equations and in many applications in signal processing.

Multiplication of a Toeplitz matrix by a vector represents (part of) a
{\em convolution}; and afficionados of Fourier analysis and signal processing
may already know that this implies that matrix multiplication can be done
in $O(n \log n)$ time using a discrete Fourier transforms.  The trick to
this is to view the Toeplitz matrix as a block in a larger {\em circulant}
matrix
\[
C =
\begin{bmatrix}
  \color{red}{a_0   } & \color{red}{a_1   } & \color{red}{a_2   } & \color{red}{a_3} & a_{-3} & a_{-2} & a_{-1}\\
  \color{red}{a_{-1}} & \color{red}{a_0   } & \color{red}{a_1   } & \color{red}{a_2} & a_{3} & a_{-3} & a_{-2}\\
  \color{red}{a_{-2}} & \color{red}{a_{-1}} & \color{red}{a_0   } & \color{red}{a_1} & a_{2} & a_{3} & a_{-3} \\
  \color{red}{a_{-3}} & \color{red}{a_{-2}} & \color{red}{a_{-1}} & \color{red}{a_0} & a_{1} & a_{2} & a_{3} \\
  a_3 & a_{-3} & a_{-2} & a_{-1} & a_0 & a_1 & a_2 \\
  a_2 & a_3 & a_{-3} & a_{-2} & a_{-1} & a_0 & a_1 \\
  a_1 & a_2 & a_3 & a_{-3} & a_{-2} & a_{-1} & a_0 \\
\end{bmatrix} =
\sum_{k=-3}^3 a_{-k} P^k,
\]
where $P$ is the cyclic permutation matrix
\[
  P =
  \begin{bmatrix}
  0 & 0 & \dots & 0 & 1 \\
  1 &   &       &   & 0 \\
    & 1 &       &   & 0 \\
    &   & \ddots&   & \vdots \\
    &   &       & 1 & 0
  \end{bmatrix}.
\]
As we will see later in the course, the discrete Fourier transform matrix
is the eigenvector matrix for this cyclic permutation, and this is a
gateway to fast matrix-vector multiplication algorithms.

Closely-related to Toeplitz matrices are {\em Hankel} matrices, which
are constant on skew-diagonals (that is, they are Toeplitz matrices
flipped upside down).  Hankel matrices appear in numerous
applications in control theory.

\subsection{Separability and Kronecker product structure}

The {\em Kronecker product} $A \kron B \in \bbR^{(mp) \times (nq)}$
of matrices $A \in \bbR^{m \times n}$ and $B \in \bbR^{p \times q}$
is the (gigantic) matrix
\[
  A \kron B =
  \begin{bmatrix}
    a_{11} B & a_{12} B & \ldots & a_{1n} B \\
    a_{21} B & a_{22} B & \ldots & a_{2n} B \\
    \vdots & \vdots & & \vdots \\
    a_{m1} B & a_{m2} B & \ldots & a_{mn} B
  \end{bmatrix}.
\]
Multiplication of a vector by a Kronecker product represents
a matrix triple product:
\[
  (A \kron B) \vec(X) = \vec(BXA^T)
\]
where $\vec(X)$ represents the vector formed by listing the
elements of a matrix in column major order, e.g.
\[
  \vec \begin{bmatrix} 1 & 3 \\ 2 & 4 \end{bmatrix} =
  \begin{bmatrix} 1 \\ 2 \\ 3 \\ 4 \end{bmatrix}.
\]

Kronecker product structure appears often in control theory
applications and in problems that arise from difference or
differential equations posed on regular grids --- you should
expect to see it for regular discretizations of differential
equations where separation of variables works well.  There is
also a small industry of people working on {\em tensor decompositions},
which feature sums of Kronecker products.

\subsection{Low-rank block structure}

In problems that come from certain areas of mathematical physics,
integral equations, and PDE theory, one encounters matrices that are not
low rank, but have  {\em low-rank submatrices}.  The {\em fast multipole
method} computes a matrix-vector product for one such class of matrices;
and again, there is a cottage industry of related methods, including
the $\mathcal{H}$ matrices studied by Hackbush and colleagues, the
sequentially semi-separable (SSS) and heirarchically
semi-separable (HSS) matrices, quasi-separable matrices, and a horde of
others.  A good reference is the pair of books by Vandebril, Van Barel
and Mastronardi~\cite{Vandebril:2010:Linear,Vandebril:2010:Eigen}.

\bibliography{refs}
\bibliographystyle{plain}

\end{document}
